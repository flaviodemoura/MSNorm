\documentclass[a4paper]{llncs}
\usepackage[utf8]{inputenc}
\usepackage[T1]{fontenc}
\usepackage[british]{babel}
\usepackage{lmodern}
%\usepackage{fullpage}
\usepackage{coqdoc}
\usepackage{amsmath,amssymb}
\usepackage{amsfonts, color}
\usepackage{url}
\usepackage{hyperref}
\usepackage[backend=bibtex,style=numeric-comp,sorting=none,firstinits=true]{biblatex}
\bibliography{MSNorm.bib}

\newcommand{\flavio}[1]{{\color{red} #1}}

\title{Formalising a constructive proof of the Modular Strong Normalisation Theorem}
\author{TBD}

\makeindex

\begin{document}

\maketitle

\begin{abstract}
  Modularity is a desirable property of rewriting systems because it allows a combined system to inherit the properties of its components. Termination is not modular, nevertheless under certain restrictions modularity can be recovered. In this work we present a formalisation of a proof of a modular strong normalisation theorem in the Coq proof assistant. The formalised proof is constructive unlikely the standard technique used in most strong normalising proofs.
\end{abstract}

\thispagestyle{empty}
\mbox{}\vfill

%\phantomsection
%\tableofcontents

\documentclass[a4paper]{llncs}
\usepackage[utf8]{inputenc}
\usepackage[T1]{fontenc}
\usepackage[british]{babel}
\usepackage{lmodern}
%\usepackage{fullpage}
\usepackage{coqdoc}
\usepackage{amsmath,amssymb}
\usepackage{amsfonts, color}
\usepackage{url}
\usepackage{hyperref}
\usepackage[backend=bibtex,style=numeric-comp,sorting=none,firstinits=true]{biblatex}
\bibliography{MSNorm.bib}

\newcommand{\flavio}[1]{{\color{red} #1}}

\title{Formalising a constructive proof of the Modular Strong Normalisation Theorem}
\author{TBD}

\makeindex

\begin{document}

\maketitle

\begin{abstract}
  Modularity is a desirable property of rewriting systems because it allows a combined system to inherit the properties of its components. Termination is not modular, nevertheless under certain restrictions modularity can be recovered. In this work we present a formalisation of a proof of a modular strong normalisation theorem in the Coq proof assistant. The formalised proof is constructive unlikely the standard technique used in most strong normalising proofs.
\end{abstract}

\thispagestyle{empty}
\mbox{}\vfill

%\phantomsection
%\tableofcontents

\documentclass[a4paper]{llncs}
\usepackage[utf8]{inputenc}
\usepackage[T1]{fontenc}
\usepackage[british]{babel}
\usepackage{lmodern}
%\usepackage{fullpage}
\usepackage{coqdoc}
\usepackage{amsmath,amssymb}
\usepackage{amsfonts, color}
\usepackage{url}
\usepackage{hyperref}
\usepackage[backend=bibtex,style=numeric-comp,sorting=none,firstinits=true]{biblatex}
\bibliography{MSNorm.bib}

\newcommand{\flavio}[1]{{\color{red} #1}}

\title{Formalising a constructive proof of the Modular Strong Normalisation Theorem}
\author{TBD}

\makeindex

\begin{document}

\maketitle

\begin{abstract}
  Modularity is a desirable property of rewriting systems because it allows a combined system to inherit the properties of its components. Termination is not modular, nevertheless under certain restrictions modularity can be recovered. In this work we present a formalisation of a proof of a modular strong normalisation theorem in the Coq proof assistant. The formalised proof is constructive unlikely the standard technique used in most strong normalising proofs.
\end{abstract}

\thispagestyle{empty}
\mbox{}\vfill

%\phantomsection
%\tableofcontents

\documentclass[a4paper]{llncs}
\usepackage[utf8]{inputenc}
\usepackage[T1]{fontenc}
\usepackage[british]{babel}
\usepackage{lmodern}
%\usepackage{fullpage}
\usepackage{coqdoc}
\usepackage{amsmath,amssymb}
\usepackage{amsfonts, color}
\usepackage{url}
\usepackage{hyperref}
\usepackage[backend=bibtex,style=numeric-comp,sorting=none,firstinits=true]{biblatex}
\bibliography{MSNorm.bib}

\newcommand{\flavio}[1]{{\color{red} #1}}

\title{Formalising a constructive proof of the Modular Strong Normalisation Theorem}
\author{TBD}

\makeindex

\begin{document}

\maketitle

\begin{abstract}
  Modularity is a desirable property of rewriting systems because it allows a combined system to inherit the properties of its components. Termination is not modular, nevertheless under certain restrictions modularity can be recovered. In this work we present a formalisation of a proof of a modular strong normalisation theorem in the Coq proof assistant. The formalised proof is constructive unlikely the standard technique used in most strong normalising proofs.
\end{abstract}

\thispagestyle{empty}
\mbox{}\vfill

%\phantomsection
%\tableofcontents

\include{MSNorm.v}

%\clearpage
%\addcontentsline{toc}{chapter}{Bibliography}
%\bibliographystyle{plain}
%\bibliography{MSNorm.bib}

% \clearpage
% \addcontentsline{toc}{chapter}{Index}
% \printindex

\printbibliography

\end{document}


%\clearpage
%\addcontentsline{toc}{chapter}{Bibliography}
%\bibliographystyle{plain}
%\bibliography{MSNorm.bib}

% \clearpage
% \addcontentsline{toc}{chapter}{Index}
% \printindex

\printbibliography

\end{document}


%\clearpage
%\addcontentsline{toc}{chapter}{Bibliography}
%\bibliographystyle{plain}
%\bibliography{MSNorm.bib}

% \clearpage
% \addcontentsline{toc}{chapter}{Index}
% \printindex

\printbibliography

\end{document}


%\clearpage
%\addcontentsline{toc}{chapter}{Bibliography}
%\bibliographystyle{plain}
%\bibliography{MSNorm.bib}

% \clearpage
% \addcontentsline{toc}{chapter}{Index}
% \printindex

\printbibliography

\end{document}
