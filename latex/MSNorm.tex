\documentclass[a4paper,envcountsame]{llncs}
\usepackage[utf8]{inputenc}
\usepackage[T1]{fontenc}
\usepackage[british]{babel}
\usepackage{lmodern}
%\usepackage{fullpage}
\usepackage{coqdoc}
\usepackage{amsmath,amssymb}
\usepackage{amsfonts, color}
\usepackage{url}
\usepackage{alltt}
\usepackage{hyperref}
\usepackage{graphicx}
\usepackage[backend=bibtex,style=numeric-comp,sorting=none,firstinits=true]{biblatex}
\usepackage{tikz}
\usetikzlibrary{positioning,shapes,fit,arrows}

\definecolor{myblue}{RGB}{56,94,141}

\bibliography{MSNorm.bib}


% \newcommand{\flavio}[1]{{\color{red} #1}}
% \newcommand{\cflavio}[2]{{\color{red} {#2}}}
% \newcommand{\raphael}[1]{{\color{blue} #1}}
% \newcommand{\craphael}[2]{{\underline{#1}}{\color{blue} {(#2)}}}
% \newcommand{\dan}[1]{{\color{brown} #1}}
% \newcommand{\cdan}[2]{{\color{brown} {#2}}}
% \newcommand{\fab}[1]{{\color{magenta} #1}}
% \newcommand{\cfab}[2]{{\color{magenta} {#2}}}
% \usepackage{ulem}
% \newcommand{\odan}[1]{\text{\sout{\ensuremath{\dan{#1}}}}}


\title{A constructive formalisation of the Modular Strong Normalisation Theorem}
\author{Flávio L. C. de Moura\inst{1} \and Daniel L. Ventura\inst{2} \and
Raphael S. Ramos\inst{1} \and Fabrício S. Paranhos\inst{2}}

\institute{Departamento de Ciência da Computação, Universidade de Brasília, Brazil\\
\email{flaviomoura@unb.br,raphael.soares.1996@gmail.com}
\and
Instituto de Informática, Universidade Federal de Goiás, Brazil \\
\email{daniel@inf.ufg.br,paranhos.s.f@gmail.com}}

\makeindex

\begin{document}

\maketitle

\begin{abstract}
  Modularity is a desirable property of rewrite systems because it
  allows a combined system to inherit the properties of its
  components. Termination is not modular, nevertheless under certain
  restrictions modularity can be recovered. In this work we present a
  formalisation of the Modular Strong Normalisation Theorem in the Coq
  proof assistant. The formalised proof is constructive in the sense
  that it does not rely on classical logic, which is interesting from
  the computational point of view due to the corresponding algorithmic
  content of proofs.
\end{abstract}
\documentclass[12pt]{llncs}
\usepackage[]{inputenc}
\usepackage[T1]{fontenc}
\usepackage{lmodern}
%\usepackage{fullpage}
\usepackage{coqdoc}
\usepackage{amsmath,amssymb}
\usepackage{url}
\usepackage{hyperref}

\title{Formalising a constructive proof of the Modular Strong Normalisation Theorem}
\author{TBD}

\makeindex

\begin{document}

\maketitle

\begin{abstract}
This report presents a formalisation of the modular strong normalisation theorem.
\end{abstract}

\begin{center} \url{http://flaviomoura.mat.br} \end{center}

\thispagestyle{empty}
\mbox{}\vfill

%\phantomsection
%\tableofcontents

\include{Intro.v}
\include{NormalisationTheory.v}
\documentclass[12pt]{llncs}
\usepackage[]{inputenc}
\usepackage[T1]{fontenc}
\usepackage{lmodern}
%\usepackage{fullpage}
\usepackage{coqdoc}
\usepackage{amsmath,amssymb}
\usepackage{url}
\usepackage{hyperref}

\title{Formalising a constructive proof of the Modular Strong Normalisation Theorem}
\author{TBD}

\makeindex

\begin{document}

\maketitle

\begin{abstract}
This report presents a formalisation of the modular strong normalisation theorem.
\end{abstract}

\begin{center} \url{http://flaviomoura.mat.br} \end{center}

\thispagestyle{empty}
\mbox{}\vfill

%\phantomsection
%\tableofcontents

\include{Intro.v}
\include{NormalisationTheory.v}
\documentclass[12pt]{llncs}
\usepackage[]{inputenc}
\usepackage[T1]{fontenc}
\usepackage{lmodern}
%\usepackage{fullpage}
\usepackage{coqdoc}
\usepackage{amsmath,amssymb}
\usepackage{url}
\usepackage{hyperref}

\title{Formalising a constructive proof of the Modular Strong Normalisation Theorem}
\author{TBD}

\makeindex

\begin{document}

\maketitle

\begin{abstract}
This report presents a formalisation of the modular strong normalisation theorem.
\end{abstract}

\begin{center} \url{http://flaviomoura.mat.br} \end{center}

\thispagestyle{empty}
\mbox{}\vfill

%\phantomsection
%\tableofcontents

\include{Intro.v}
\include{NormalisationTheory.v}
\include{MSNorm.v}
\include{Conclusion.v}

\clearpage
%\addcontentsline{toc}{chapter}{Bibliography}
\bibliographystyle{plain}
%\bibliography{/homedir/Dropbox/bibliography/references.bib}

% \clearpage
% \addcontentsline{toc}{chapter}{Index}
% \printindex

\end{document}

\include{Conclusion.v}

\clearpage
%\addcontentsline{toc}{chapter}{Bibliography}
\bibliographystyle{plain}
%\bibliography{/homedir/Dropbox/bibliography/references.bib}

% \clearpage
% \addcontentsline{toc}{chapter}{Index}
% \printindex

\end{document}

\include{Conclusion.v}

\clearpage
%\addcontentsline{toc}{chapter}{Bibliography}
\bibliographystyle{plain}
%\bibliography{/homedir/Dropbox/bibliography/references.bib}

% \clearpage
% \addcontentsline{toc}{chapter}{Index}
% \printindex

\end{document}


%\clearpage
%\addcontentsline{toc}{chapter}{Bibliography}
%\bibliographystyle{plain}
%\bibliography{MSNorm.bib}

% \clearpage
% \addcontentsline{toc}{chapter}{Index}
% \printindex

\renewcommand{\em}{\it}
\printbibliography

\end{document}
